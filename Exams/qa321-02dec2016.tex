\documentclass[oneside]{article}
\title{Quiz}
\author{Orgho Neogi}
\date{02 December 2016}

\usepackage{amsmath}

\newenvironment{answer}
  {\vspace*{0.2cm} \rule{12cm}{0.02cm} \vspace*{0.2cm}}
  {\vspace*{0.2cm}}


\begin{document}
\maketitle

\begin{enumerate}
  \item We are working in three dimensions but we are
    using vectors that contain four elements (not three)
    and matrices that contain $4 \times 4$ elements.
    What have we gained by choosing $4 \times 4$ matrices
    rather than $3 \times 3$ matrices?

  \begin{answer}

  We have gained the ability to apply multiple transformationsafter one another using the same matrix.
  If we use $4 \times 4$ matrices, then we gain the ability to represent translation using a matrix multiplication
  which can't be done in a 3D system with $3 \times 3$ matrices. This means that for a complex transformation, the computationally
  intensive matrix multiplication can be done just once to get the matrix that we need that represents the transformation
  and then multiplied with all the relevant vectors. Using $4 \times 4$ matrices makes the perspective projection
  expressible with a matrix too.

    \end{answer}

  \item Many systems (including three.js, the system that we have
    chosen) support both RGB and HSL color. Distinguish between
    the two. How does a programmer specify a color using the
    two systems?

  \begin{answer}

  An RGB color system works by defining a color by specifying the intensity levels of the colors red, green, and blue.
  A HSL color system works by defining a color by specifying the hue, saturation, and lightness.
  Each is a model of working with colors and a different way to specify a color.
  The advantage of HSL is that the transition from black to a hue to white is symmetric and is controlled solely by increasing lightness.
  However, RGB directly reflects how a Truecolor display renders colors.

    \end{answer}

  \item Programmers can specify a cubic B\'{e}zier curve by
    providing four points.
  \begin{enumerate}
    \item Which of the points will lie on the curve?
    \item The curve will be contained in the convex hull
      of the four points. What is the convex hull?
    \item Let $\vec{p_0}, \vec{p_1}, \vec{p_2}$ and $\vec{p_3}$
      be vectors that specify the location
      of the four points that specify a cubic B\'{e}zier curve.
      What is the geometric significance of $\vec{p_1} - \vec{p_0}$
      and $\vec{p_3} - \vec{p_2}$?
    \item How can a programmer rotate a B\'{e}zier curve?
    \end{enumerate}

  \begin{answer}

  \begin{enumerate}
    \item The first and the last point specified will lie on the curve. The others usually do not.
    \item The convex hull is the smallest convex set that contains the set of points in question.
    A convex set is a region such that for every pair of points within the region, every point on
    the straight line segment that joins the pair of points is also within the region.
    This means that a B\'{e}zier curve will always be inside it's convex hull.
    \item The tangent to the curve at point $p_0$ is in the same direction as the vector $\vec{p_1} - \vec{p_0}$ and
    the tangent to the curve at point $p_3$ is in the same direction as the vector $\vec{p_3} - \vec{p_2}$. This means that
    the derivative at the first or last point of a B\'{e}zier curve is the same as the gradient of the line connecting the first and
    second point or the second last and last point respectively.
    \item To rotate a B\'{e}zier curve, the rotation matrix can be multiplied with the control points to rotate the entire B\'{e}zier curve.
    \end{enumerate}

    \end{answer}

  \item The JavaScript language allows programmers to assign
    functions to variables, to pass a function $a$ to another function $b$,
    and to return a function $d$ from another function $c$ to $c$'s caller.
    This is a powerful feature of the language. What is this feature called?

  \begin{answer}

    JS supports functional programing because you can do the things mentioned above.

    \end{answer}

  \item Let $\vec{p_0}, \vec{p_1}$ and $\vec{p_2}$ be vectors that
    describe the location of the three vertices of a triangle.

    What is the geometric signficance of the following expression?

  \begin{align*}
    \frac{1}{\mid (\vec{p_1} - \vec{p_0}) \times (\vec{p_2} - \vec{p_0}) \mid}
    \;\;
    ((\vec{p_1} - \vec{p_0}) \times (\vec{p_2} - \vec{p_0}))
    \end{align*}

  \begin{answer}

    The expression will give the normal unit vector to any point on the triangle. This is what is
    used to calculate shading of the triangle in a 3D renderer.

    \end{answer}

  \item Let $\hat{n}$ and $\hat{s}$ be two vectors. The magnitude
    of each vector is one. The vector $\hat{n}$ is normal (perpendicular)
    to some small piece of a surface in our virtual world.
    The vector $\hat{s}$ points toward the source of light.

    What is the signficance of the dot product of these two vectors?

  \begin{answer}

    The dot product is used as a percentage of the amount of light that strikes
    the polygon region and thus can be used to emulate the effects of light and shadow.

    \end{answer}

  \end{enumerate}


\end{document}
